%
% Acknoledgements: these slides are largely based on Gregor Dick's examples!
%

\documentclass[12pt]{beamer}

\usetheme{Oxygen}
\usepackage{thumbpdf}
\usepackage{wasysym}
\DeclareGraphicsRule{*}{mps}{*}{}
\usepackage{ucs}
\usepackage[utf8]{inputenc}
\usepackage{pgf,pgfarrows,pgfnodes,pgfautomata,pgfheaps,pgfshade}
\usepackage{verbatim}
\usepackage{pgfpages}
\usepackage{listings}
\usepackage{fix-cm}

\let\Tiny=\tiny

\lstset{
	language=C++,
	keywordstyle=\bfseries\ttfamily\color[rgb]{0,0,1},
	identifierstyle=\ttfamily,
	commentstyle=\color[rgb]{0.133,0.545,0.133}\itshape,
	stringstyle=\ttfamily\color[rgb]{0.627,0.126,0.941},
	showstringspaces=false,
	basicstyle=\small\ttfamily,
	numberstyle=\footnotesize,
	numbers=none,
	stepnumber=1,
	numbersep=10pt,
	tabsize=2,
	breaklines=true,
	prebreak = \raisebox{0ex}[0ex][0ex]{\ensuremath{\hookleftarrow}},
	breakatwhitespace=false,
	aboveskip={1.5\baselineskip},
	columns=fixed,
	%upquote=true,
	extendedchars=true,
	% frame=single,
	backgroundcolor=\color[rgb]{0.9,0.9,0.9},
	xleftmargin=2em,
	xrightmargin=2em,
}

\addtolength{\arraycolsep}{-3pt}

\definecolor{myblue}{rgb}{0.25, 0.25, 0.75}
\newcommand{\ct}[1]{\color{myblue} #1\color{black}}% Color text
\newcommand{\ctt}[1]{\color{red} #1\color{black}}% Color text
\newcommand{\myref}[1]{\ct{{\small[#1]}}}
\newcommand{\mycode}[1]{\texttt{\ctt{#1}}}

\newcommand{\be}{\begin{equation}}
\newcommand{\ee}{\end{equation}}
\newcommand{\om}{\omega}
\newcommand{\vp}{\varphi}
\newcommand{\ra}{\rightarrow}
\newcommand{\D}{\text{d}}
\newcommand{\I}{\text{i}}
\newcommand{\E}{\text{e}}
\newcommand{\paper}{paper\ }
\newcommand{\lpole}{\times}
\newcommand{\lsp}{*}
\newcommand{\mE}{\mathcal{E}}
\newcommand{\mU}{\mathcal{U}}
\newcommand{\reals}{\mathbb{R}}


\pdfinfo
{
  /Title       (C/C++ programming Functions and Pointers)
  /Creator     (TeX)
  /Author      (Massimo Cavallaro)
}


\title[C++]{Programming seminar 2015 \\ Session 4: Functions and Pointers}
%\subtitle{steady state and fluctuations}
\author{Massimo Cavallaro}
\institute[QMUL]{School of Mathematical Sciences\\
Queen Mary, University of London}
\date[11 March 2015]{11 March 2015}

\begin{document}

\frame{\titlepage}

\section*{}
\begin{frame}
  \frametitle{Outline}
  \tableofcontents[section=1,hidesubsections]
\end{frame}

\AtBeginSection[]
{
  \frame<handout:0>
  {
    \frametitle{Outline}
    \tableofcontents[currentsection,hideallsubsections]
  }
}

\AtBeginSubsection[]
{
  \frame<handout:0>
  {
    \frametitle{Outline}
    \tableofcontents[sectionstyle=show/hide,subsectionstyle=show/shaded/hide]
  }
}

\newcommand<>{\highlighton}[1]{%
  \alt#2{\structure{#1}}{{#1}}
}

\newcommand{\icon}[1]{\pgfimage[height=1em]{#1}}

% % % % % 
% CONTENT STARTS HERE
%

\section{Calling functions}

\begin{frame}
\frametitle{Motivation}
%\framesubtitle{Divide}

Real-world computer programs are (much) larger than those seen so far.
The strategy to develop, maintain and understand large programs is to use small components.
A \textit{function} is one of these components. 
It is a block of code that typically takes parameters and returns a
value. 
\end{frame}


\defverbatim[colored]\MathList{%
\begin{lstlisting}
exp(x)
cos(x)
pow(x)
fabs(x)
floor(x)
ceil(x)  
fmod(x)  
\end{lstlisting}}
\defverbatim[colored]\MathHeader{%
\begin{lstlisting}
#include <cmath>
\end{lstlisting}}
\begin{frame}
  \frametitle{Prototypes}
\framesubtitle{Example: Math library functions}
You are already familiar with some functions, e.g.,
 \MathList
These functions perform a precise task and can be inserted in any part of the  program.
To use them you need:
 \MathHeader
But why?
\end{frame}

\defverbatim[colored]\LstProto{%
\begin{lstlisting}[emph={condition},emphstyle=\emph]
double sin(double angle);
double pow(double base, double exponent);
void my_function(int a, char b, string c);
\end{lstlisting}}
\begin{frame}
  \frametitle{Prototypes}
%  \framesubtitle{}
	Every function --- whether one we've written ourselves or one we're
	using from somewhere else --- must have a prototype before it can be
	called. A prototype describes the type of the value (if any) returned
	by the function, and the types of all its parameters. Here are some
	example prototypes:
	\LstProto
	The prototype should appear at the top level, and before the function
	is called.
\end{frame}


\begin{frame}
  \frametitle{Prototypes}
%  \framesubtitle{}
	If we are using a library function, we typically get its prototype by
	including the appropriate \textbf{header file} for the library (this is
	what, e.g., \lstinline/\#include <cmath>/ does: it takes the contents of
	the system header file \lstinline/cmath/, in which are the prototypes
	for the standard mathematical functions, and adds it to the current
	file).
\end{frame}



\begin{frame}
  \frametitle{C++ Standard Library header files}
  \framesubtitle{Not only mathematical functions...}
   The following header files contain the prototypes for some of the functions you used in the last tutorials,
   They contain also definitions of types and constants need by those functions:
   \begin{itemize}
    \item \lstinline/\<iostream>/ def. the standard input/output stream objects (\lstinline/std::cout, std::cin/)
    \item \lstinline/\<cstdlib>/  def. the random number generator ( \lstinline/rand()/ )
    \item \lstinline/\<cstdio>/   def. the c-style input/output functions ( \lstinline/printf(const char *format, ...)/ )
    \item \lstinline/\<ctime>/    Contains func. prototypes and types for manipulating the time and date.
    \item \lstinline/\<cstdarg>/  Support for the variadic function (last slide).
   \end{itemize}
   For a complete list see, e.g., \href{http://www.cplusplus.com/reference/}{http://www.cplusplus.com/reference/}
\end{frame}


\defverbatim[colored]\LstCall{%
\begin{lstlisting}
double some_sine = sin(100);
printf("Hello, world!\n");
int random_num = rand();
\end{lstlisting}}
\begin{frame}
	\frametitle{Calling functions}
	We call a function by writing its name, followed by brackets containing
	the (actual) parameters (i.e.\ the values the parameters should take),
	separated by commas; if there are no parameters, the brackets are left
	empty. For example:
	\LstCall
    See \lstinline/functions.cpp/
\end{frame}

\section{Writing functions}

\defverbatim[colored]\LstFunction{%
\begin{lstlisting}
int some_function(int n){
	if(n <= 0) return 0;
	cout << "Input was positive." << endl;
	return 100 * n;
}
\end{lstlisting}}
\begin{frame}
	\frametitle{Writing our own functions}
	The first line of a function is the same as its prototype, except that
	it doesn't end in a semicolon and must contain the argument names. The \textbf{body} of the function follows
	--- that is, the code that is to be executed when the function is
	called. The \lstinline/return/ statement is used to end the function and
	return a value.
	\LstFunction
    See \lstinline/myfirstfunction.cpp/.
\end{frame}

\defverbatim[colored]\LstScope{%
\begin{lstlisting}
int main() {
	int i;
	cout << fn() << endl;
	return 0;
}
int fn() {
	return i * 2;	// ERROR
}
\end{lstlisting}}
\begin{frame}
	\frametitle{Variables in functions: scope}
	Variables declared inside a function are said to be \textbf{local} to
	that function --- that is, they're visible within that function only. So
	this is an error:
	\LstScope
	To make a variable visible to all functions --- we say, \textbf{global}
	--- define it at the top level, outside any function. But avoid this if
	possible: use parameters instead.
    See \lstinline/glocal.cpp/
\end{frame}



\defverbatim[colored]\LstParamPass{%
\begin{lstlisting}
int main() {
	int i = 10;
	mystery(i);
	cout << "i is " << i << endl;
	return 0;
}
void mystery(int n) {
	n++;
	cout << "n is " << n << endl;
}
\end{lstlisting}}
\begin{frame}
	\frametitle{Parameter-passing in detail}
	What is the output when running the following?
	\LstParamPass
    Copy/Paste this snippet and compile. Hint: It's nothing to do with the names of the variables!
\end{frame}


\begin{frame}
	\frametitle{Parameters are passed by value}
	When we call a function, \emph{copies of the values} of the actual
	parameters are made for use by the function, so changes to the
	parameters within the function have no effect on the caller. This
	arrangement is known as \textbf{passing by value}.

    We can do more.

    But first...
    
\end{frame}


\begin{frame}
	\frametitle{Exercises!}
	\framesubtitle{based on the last tutorial!}
    \begin{itemize}
        \item Write a function that prints $n$ times \lstinline/hello world/
        \item Write a function that calculates and returns the factorial of $n$. See \lstinline/forloop3factorial.cpp/
        \item Write a function \lstinline/mypow(x,n)/ that calculates and returns $x^n$, $n \in \mathbb{N}$. Compare its speed with those of \lstinline/pow(x,n)/.
        \item Write a function that calculates and returns the exponential of $x$ with a certain precision $prec$. Make use of \lstinline/mypow(x,n)/.
             See \lstinline/whileloopexponential.cpp/
    \end{itemize}
\end{frame}


\section{References and pointers}

\defverbatim[colored]\LstRef{%
\begin{lstlisting}
void mystery(int& n){
	n++;
	cout << "n is " << n << endl;
}
\end{lstlisting}}
\begin{frame}
\frametitle{Pass by reference}
	To get the opposite behaviour, whereby a copy is \emph{not} made, and
	the function operates on an actual parameter directly, so that changes
	to that parameter \emph{do} affect the caller, specify that that
	parameter is a \textbf{reference}, which is achieved by adding an
	ampersand to the end of the type (both in definition and prototype), e.g.
	\LstRef
	This is called \textbf{passing by reference}.
    How does it works?
\end{frame}


\begin{frame}
\frametitle{Pass by reference 2}

\begin{block}{Example}
Write a function that double the value in the caller,
and returns void. see \lstinline/ref.cpp/
\end{block}

    Roughly speaking, the ``ampersand'' \lstinline/&/ means ``address of''.
    In the previous snippet, the address of the variable \lstinline/n/ is copied and passed. That address can't change.
    But, after, the function works directly on the memory with address \lstinline/&n/.
\end{frame}

\defverbatim[colored]\AnsiCPoint{%
\begin{lstlisting}
#include <stdio.h>
void modify(int* c){
    *c = 6;
}
int main(){
    int d = 10;
    modify(&d);
    printf("%d\n",d);
    return 1;
}
\end{lstlisting}}
% //*c dereferenced pointer or Indirection ("object pointed to by a")
\begin{frame}
\frametitle{Pass by reference 3}
\framesubtitle{ANSI C is syntactically more coherent}
    You can't pass by reference and you must use the \textit{dereference operator} \lstinline/*/ in order to pass an address.
    \AnsiCPoint
    In function \lstinline/modify/, \lstinline/c/ is a pointer to an integer while \lstinline/*c/ is an integer.
    See \lstinline/pointer.c/
\end{frame}



\section{Variable number of arguments}

\defverbatim[colored]\VaList{%
\begin{lstlisting}
double ArithmeticMean(int num,...){
    va_list valist; //declaration
    double sum = 0.0;
    int i;
    va_start(valist, num); //initialization of valist
    for (i = 0; i < num; i++)
       sum += va_arg(valist, int);   //access
    va_end(valist);
    return sum/num;
}
\end{lstlisting}}
\begin{frame}
\frametitle{Variable number of arguments}
If we include the header \lstinline/cstdarg/, we have access to some useful functions that allow us to
define function with a variable number of arguments. See \lstinline/variadic.cpp/.


\VaList


\end{frame}



\end{document}
